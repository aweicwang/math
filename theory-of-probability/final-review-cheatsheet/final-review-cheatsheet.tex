\documentclass{article}[11pt]

\usepackage{amssymb}
\usepackage{amsmath}
\usepackage[margin=1in]{geometry}
\usepackage[utf8]{inputenc}
\setlength{\parindent}{0pt}
\setlength{\parskip}{1em}


\title{Final Review}
\author{Andrew Wang}

\begin{document}

\maketitle

\section{Counting}

\textbf{Theorem. (Binomial Theorem)} Let $x,y \in \mathbb{R}$ and $n \in \mathbb{N}$. Then
\[
  (x+y)^n = \sum_{0 \leq i \leq n}\binom{n}{i}x^iy^{n-i}.
\]
  
\textbf{Theorem. (Multinomial Theorem)} Let $a_1, \dots, a_n \in \mathbb{R}$ and $n \in \mathbb{N}$. Then
\[
  (a_1 + \dots + a_k)^n = \sum_{m_1 + \dots + m_k = n}\binom{n}{m_1, \dots m_k}a_1^{m_1}\dots a_n^{m_k}.
\]

\section{Probabilities}

\textbf{Definition.} The \emph{conditional probability} of $E$ given $F$ is

$$\mathbb{P}(E|F) = \frac{\mathbb{P}(E\cap F)}{\mathbb{P}(F)}$$

\textbf{Definition.} Two events $E$ and $F$ are said to be \emph{independent} if
$$\mathbb{P}(E\cap F) = \mathbb{P}(E)\mathbb{P}(F).$$

\textbf{Theorem. (Bayes' Formula)} $$\mathbb{P}(E|F)=\frac{\mathbb{P}(F|E)\mathbb{P}(E)}{\mathbb{P}(F)}.$$


\section{Random Variables}

\textbf{Definition.} The \emph{variance} of a real valued random variable $X$ is given by
$$Var(X) = \mathbb{E}[X^2] - \mathbb{E}[X]^2 = \mathbb{E}[(X-\mathbb{E}[X])^2].$$


\textbf{Definition.} Two random variables $X, Y$ are said to be independent if 
$$\mathbb{P}(X=a,Y=b) = \mathbb{P}(X=a)\mathbb{P}(Y =b)$$
for all $a,b$.

\textbf{Proposition.} Let $X, Y$ be two independent random variables. then: \\
(a) $g(X), f(Y)$ are independent for all functions $g,f$ on $X, Y$.\\
(b) $\mathbb{E}[g(X)h(Y)] = \mathbb{E}[g(X)] \mathbb{E}[f(Y).]$\\
(c) $Var(g(X) + f(Y)) = Var(g(X)) + Var(f(Y))$

\section{Poisson Random Variables}

\textbf{Definition.} $X$ is a \emph{binomial random variable} $X$ with parameter $p \in (0,1)$ and $n \in \mathbb{N}$
is given by $$\mathbb{P}(X = k) = \binom{n}{k}p^k(1-p)^{n-k}$$ for some $k \in \{1,2,\dots, n\}.$
\section{Moment Generation Functions}

\textbf{Definition.} For a real-valued random variable $X$, its \emph{moment-generating function}
is $M_X: \mathbb{R}\to [0, \infty]$, $$M_X(t) = \mathbb{E}[e^{tX}], t \in \mathbb{R}$$

\par

\textbf{Proposition.} \\ 
(a) $M_X(0) = 1$. \\
(b) $M_{X+Y} = M_XM_Y$ for $X,Y$ independent, \\
(c) $M_X$ is fully determined by the moments of $X$ so long as 
$\mathbb{E}[X^n]\leq(cn)^n$ for some $c >0$.


\par
\textbf{Definition.} For random variables $X_1, \dots, X_n$, they have moment generating function
$M_{X_{1, \dots, n}}(t_1, \dots, t_n) = \mathbb{E}[e^{t_1X_1 + \dots + t_nX_n}].$

\par

\textbf{Definition.} $X_1, X_2, \dots, X_n$ are \emph{multivariate normal random variable} if for any $X_i \in {X_1, \dots, X_n}$
                     there exist independent normal random variables $Y_1, Y_2, \dots, Y_m$ and $a_{ij} \in \mathbb{R}$ such that
                     $$\sum_{0\leq j \leq m}a_{ij}Y_j = X_i.$$

 
\section{Conditional Expectations and Inequaliies.}

\textbf{Definition.} For a set $S$ and $A \subset S$, the \emph{indicator/characteristic function} of $A$ is 
\[
\mathcal{X}_A(x) = \begin{cases}
1, x \in A, \\
0, x \notin. A
\end{cases}
\]

\textbf{Proposition. (Chevyshev's Inequality)} For $a > 0$ and a real-valued random variable $X$,
$$\mathbb{P}(| X - \mathbb{E}[X]| \geq a) \leq \frac{Var(X)}{a^2}.$$

\textbf{Proposition. (Markov's Inequality)} For $a > 0$, and a real-valued random variable $X \geq 0$,
$$\mathbb{P}(X \geq a) \leq \frac{\mathbb{E}[X]}{a}.$$

\textbf{Definition.} The \emph{conditional expectation} of $X$ given $Y$ is a ramdom variable $T$ with
\[
  \mathbb{P}(T = s) = \sum_{y \in Y: E[X|Y = y] = s} \mathbb{P}(Y=y)  
\]
and denoted by $\mathbb{E}[X|Y].$ Similarly, when $X,Y$ are jointly continuous,
$$\mathbb{E}[X|Y=y] = \frac{1}{f_Y(y)}\int_{\mathbb{R}}xf(x,y)dx$$
\[
  \mathbb{P}(T\in I) = \int_{S(I)}f(y)dy
\]
where $S(I) = {y\in \mathbb{R}: \mathbb{E}}[X|Y = y] \in I$

\textbf{Proposition.} \\
(a) $\mathbb{E}[X] = \mathbb{E}[\mathbb{E}[X|Y]]$, \\
(b) $\mathbb{E}[Xg(Y)|Y] = g(Y)\mathbb{E}[X|Y]$.
\section{Law of Large Numbers}

\textbf{Definition.} $X_n$ \emph{converges in probability} to $X$ if
                     for any $\varepsilon > 0$,
                     $$\lim_{n \to \infty}\mathbb{P}(|X_n-X| \geq \varepsilon) = 0 $$
                     denoted by $X_n \xrightarrow{p} X $.

\par

\textbf{Definition.} $X_n$ \emph{converges almost surely} to $X$ if 
                     $$\mathbb{P}(\lim_{n \to \infty}X_n = X) = 1.$$
                     denoted by $X_n \xrightarrow{a.s.} X.$

\textbf{Theorem. (Weak and Strong Law of Large Numbers)} Suppose $\{X_n\}$ are i.i.d random variables with $X_n \xrightarrow{d} X, \mathbb{E}[X] < \infty$, then
$$\lim_{n \to \infty}\frac{X_1 + X_2 + \dots +X_n}{n} \xrightarrow{p \ \& \ a.s.} \mathbb{E}[X].$$



\section{Central Limit Theorem}

\textbf{Definition.} $X_n\in \mathbb{R}$ \emph{converges in distribution} to $X \in \mathbb{R}$ if
                     $$\lim_{n \to \infty}\mathbb{P}(X_n \leq a) = \mathbb{P}(X \leq a)$$
                     for all $a \in \mathbb{R}$ at which the CDF of $X$ is continuous. This is denoted as $$X_n \xrightarrow{d} X.$$

                     \textbf{Theorem. (Central Limit Theorem)} Suppose $X_n \in \mathbb{R}$ are i.i.d. with $\mathbb{E}[X^2] < \infty$, $Var(X) > 0$. Then
                     $$\frac{X_1+\dots+X_n - n\mathbb{E}[X]}{\sqrt{n\cdot Var(X)}} \Rightarrow \mathcal{N}(0,1).$$


\end{document}
