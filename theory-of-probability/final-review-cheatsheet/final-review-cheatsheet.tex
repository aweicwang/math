\documentclass{article}

\usepackage{amssymb}
\usepackage{amsmath}
\usepackage[utf8]{inputenc}
\setlength{\parindent}{0pt}
\setlength{\parskip}{1em}


\title{Final Review}
\author{Andrew Wang}

\begin{document}

\maketitle
 
\section{Law of Large Numbers}

\textbf{Definition.} $X_n$ \emph{converges in probability} to $X$ if
                     for any $\varepsilon > 0$,
                     $$\lim_{n \to \infty}\mathbb{P}(|X_n-X| \geq \varepsilon) = 0 $$
                     denoted by $X_n \xrightarrow{p} X $.

\par

\textbf{Definition.} $X_n$ \emph{converges almost surely} to $X$ if 
                     $$\mathbb{P}(\lim_{n \to \infty}X_n = X) = 1.$$
                     denoted by $X_n \xrightarrow{a.s.} X.$

\textbf{Theorem.} Suppose $\{X_n\}$ are i.i.d random variables with $X_n \xrightarrow{d} X, \mathbb{E}[X] < \infty$, then
$$\lim_{n \to \infty}\frac{X_1 + X_2 + \dots +X_n}{n} \xrightarrow{p \ \& \ a.s.} \mathbb{E}[X].$$


\end{document}
